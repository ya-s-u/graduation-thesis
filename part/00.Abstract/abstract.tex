%!TEX root = ../../main.tex
\chapter*{論文要旨}
\label{chap0}
\addcontent{論文要旨}
旋律歌唱や旋律聴取における3つの処理側面として,リズム,旋律線,調性がある\cite{hatano2007}.旋律線 (pitch contour) は旋律概形とも呼ばれ,大まかな音高の上下動を指す.調性は,調やコード進行を指す.これら3要素のうち,音楽未経験者の即興合奏を難しくするのは調性判断である.調性判断とは調の種類を判定することを指す.調性以外の旋律線やリズムについては,初心者でも比較的容易に直感的動作として表出できる.そのため,音楽未経験者が即興合奏を行いたい場合,例えば地域振興のために幅広い層の自由な参加を志向した音楽イベントなどであっても,音楽未経験者は手拍子など自由度が低い手段での参加に限られる.
本研究では,ユーザの調性判断をシステムが補うことで,音楽経験に乏しい人でも調性感を損なわず自在に即興合奏に参加できるようなシステムの開発を目指す.具体的には,ユーザが旋律線とリズムを身体動作で入力すれば,システムが背景楽曲のコード進行に応じて音高補正を行い,調性の制約を満たし不協和にならない合奏ができる演奏インタフェースの実現を目指す. 
そのために,(1) 直感的な身体動作による旋律線やタイミング,音の種類などの入力手法,(2) 演奏音の出力範囲の指定手法,(3) 背景楽曲のコード進行に対する調性の制約の決定手法,の3つを課題として扱う.
(1)の直感的な入力手法としては,Intel RealSense 3D Camera (以下,RealSense) を用い,手の動きを認識して旋律線を入力する.RealSenseは手指のジェスチャー認識が容易であり,直感的な身体動作入力に適している.本研究では特に,認識誤りの多いtap動作の改善を試みた.
(2)の演奏音の出力範囲の指定手法は,手を伸ばし画面をスクロールさせるような操作モードを設け,手の深度によってこの操作モードを切り替える.
(3)の調性制約の決定手法は,能動的音楽鑑賞サービスSongle\cite{songle}のWeb APIからの解析情報を元に実現した.具体的にはSongleに公開されている100曲分について統計をとり,それをもとにコードに対する調性制約を決定した.
また,本研究で実装した改善版のtap認識,調性制約に対し,評価実験を行ったところ,tapの精度は実用的なものに向上,調性制約はコードの構成音のみと比べて意図通りの演奏がしやすく,不協和が少しあるという評価となった.