\chapter*{Abstract}
\label{chap:Abstract-e}
\addcontent{Abstract}

Because of the spread of smartphones and the growth of the network technology, use of the social networking/micro-blogging service Twitter has spread widely. 
%スマートフォンの普及やネットワーク技術の発展により,Twitter などの SNS は 急速な成長を遂げた.
Social media such as tweets is attracting attention to its information value to know what they are doing and what they are thinking. 
%個人が用意に情報を発信できるため,一般の人々が何をして いるのか,何を思っているのかなど情報源としての価値も注目されつつあるそ
In recent years, a demand of society for analyzing topics in Twitter has been increasing.
 %そのような背景に伴い,近年,Twitter における話題分析のための技術に対する社会的要請は高まってきている.
However, so far, a usage scenario where users hark back to past event through browsing past timeline of Twitter has not been considered. 
%しかしながら,従来,過去のタイムラインをアーカ イブ化して振り返るような使い方は想定されていなかった
Therefore, we proposed a method of exploratory browsing for the past tweets set about a specific topic. 
%そこで,本研究では, ある話題に関する過去のツイート群における探索的閲覧手法を提案する

For a lot of tweets,  users cannot remember contents of the past tweets. It is difficult to determine an appropriate query for searching past tweets.
%大量に存在するツイート集合において,ユーザは過去にあった大量のツイートの内容を 知るはずもなく,何をクエリとして検索すればいいのか分からない.
In order to grasp the background of a specific topic, it is desirable to know time-series flows of features of tweets about the topic such as what events make trends of tweets changed.
%特定の話題 の背景を把握するためには,その話題に関するツイートの傾向がどのような流れ で移り変わっていったか,どのような出来事がツイートの傾向を変えたかという 時系列的な流れを知ることが望ましい.
Visualizing topics using a burst detection for Twitter is one of the researches about exploratory browsing for tweets.  
%ツイートに対する探索的閲覧の研究の中 でもバースト検出を利用したトピックの可視化などがあるが,
However, the method extracts only features from tweets in the burst period. In other words, it cannot extract features from tweets in all time.   
%従来の手法では盛り上がりのあった期間の特徴のみを抽出し,それ以外の期間の特徴が欠けてしまう
Presenting  features in only a particular time span, users cannot grasp detail information such as the changes of features in the time span.
%またある固定された区間の特徴のみをユーザに提示するだけでは,その区間 内の特徴の推移など,詳細な情報を把握することができない.
For users who lack knowledge about the retrieval object, these are not satisfied about the comprehensiveness of exploratory browsing. 
%検索対象に関する 知識の乏しいユーザにとって,これらは探索的閲覧における網羅性を満たしてい ない.

To solve these problems, we implemented an exploratory tweets browsing system which can extract feature terms from each time period, in a day or in a week or in a month,  using the statistical method combining Information Gain and Pointwise Mutual Information.
%そのために,本研究では,日,週,月の異なるタイムスケール毎にツイート を分割し,それぞれの期間内ツイート群から,統計的手法である情報利得と自己 相互情報量を用いた特徴語抽出を実現,
The system enables users to browse overview of tweets changing time scale and figure out the features of tweets along time-series.
%ユーザが動的にタイムスケールを調整 しながら閲覧することで,時系列に沿ったツイートの探索が可能な探索的閲覧支 援システムを開発した
We proposed Bias-Penalized Information Gain which can exclude inadequate terms that a specific user has frequently posted. We showed the effectiveness of the method by experimental results.
%また,期間の特徴語として不適切な語を除外するための バイアス罰則付き情報利得を提案し,実験からその有効性を示した.
Moreover, we confirmed that the exploratory browsing method we proposed is effective in recognizing a causal relationship between events and  opinions in a lot of past tweets by the experiment of tracking topics using the system.
%さらに,本 システムを利用した話題追跡の実験結果から,本研究での探索的閲覧手法が,過 去の大量のツイート群に対し出来事や意見などの時系列的な因果関係の把握に有 効であることを確認した.

