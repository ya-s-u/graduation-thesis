% a sample is here.
% Mon Jan 19 2004 by 宇佐見@内匠研
% コンパイルエラーなどはこちらへ: sho@ics.nitech.ac.jp
%
\documentclass[a4j,twoside]{jarticle}
\usepackage{graphicx}
\usepackage{styles/thesis_abst}
\usepackage[dvipdfmx, usenames]{color}%%表の色付け
\usepackage{colortbl}

\newcommand{\argmax}{\mathop{\rm arg~max}\limits}

%マージンはプリンタによって変更
\addtolength{\oddsidemargin}{0mm}
\addtolength{\evensidemargin}{0mm}
\newcommand{\ve}[1]{\mbox{\boldmath$#1$}}
\def\Hline{\noalign{\hrule height 0.4mm}}
\def\halflineskip{\par\vspace{.5\baselineskip}\par}

%baselinestretchを変更すると上部枠の大きさが変わるのでおすすめしない
\renewcommand{\baselinestretch}{1}

\種別{卒 業} %{卒 業}または{修 士} 間に半角スペースを入れる
\学籍番号{24115054}
\氏名{後藤 誉昌} % 苗字と名前の間は半角スペース
%\英語氏名{} %未使用
\研究室{白松}
\題目{MissionForest: 組織内外における\\協働支援のためのタスク構造化システムの研究} % 途中で改行 "\\" を挿入可
\年度{28} % !=年 発表は2月です

\begin{document}              %この行を消してはいけない
\twocolumn[\vspace*{9mm}]     %この行を消してはいけない
\begin{論文概要}              %この行を消してはいけない
\setcounter{page}{1}          %表(左綴じ)は1,裏(右綴じ)は2を指定
%%%%%%% ここからアブスト本体 %%%%%%%
\baselineskip = 5mm

\section{はじめに}
本研究ではこれまで,公益活動やシビックテックといった分野を対象とし,公共圏で目標を共有するWebシステム「ゴオルシェア」[1]を開発・運営してきた.
従来のゴオルシェアは組織横断的な協働を想定しており,目標データを全てオープンデータ化していた.
しかし,組織内での日常的な活動は公開に適さないものも多いため,日常的には使いにくいという問題点があった.
また,目標を階層化したツリー構造の入力操作が直感的でないという問題点もあった.
そこで本稿で試作する新システム「MissionForest」では,(1)組織内部の日常的な活動を非公開な目標ツリーとして記録し,(2)外部発表後にツリー構造の一部をオープンデータ化可能にする.
さらに,(3)目標ツリーを直感的操作で作成・編集可能にする.

\section{直感的にツリーを作成できるインターフェース}
試作するシステムでは,プロジェクトを「ミッション」と呼ぶ.
ユーザーは任意にミッションを作成することができ,ミッションごとに直感的なGUIでタスクツリーを構築することができる.
タスクには進捗状況の指定,タグの指定,コメント機能,編集履歴,ファイル添付をすることができる.
作成したミッションは任意のタイミングで一般公開することができるので,公開されたミッション間から後述するアルゴリズム類似ミッションを推定し,ユーザーに推薦することにより,組織を越えた協働を促進する.

\section{LODでの公開機構}
先に述べたように,ゴオルシェアではLinked Dataのアクセス権限を指定することができず,自動的にすべてのデータが誰でも閲覧できる状況にあった.
しかし研究室など限られた組織内で使うには,アクセス権限のあるユーザーのみが閲覧できる仕組みが必要である.
そこで現状では,アカウント単位でのアクセスコントロール機能があるRDFストアであるStardogを用いて,表1に示すような,プロジェクトの段階に応じたアクセス制御機構を実装中である.
具体的には,(1)プロジェクトが個人的な構想段階である初期場合,(2)組織内限定で共有される段階,(3)外部発表後に外部公開する段階,(4)オープンライセンスで公開し,組織横断的な協働を志向する段階,の4段階を想定し,各段階に応じたアクセス制御機構を想定している.
プロジェクトの段階が進んでいくにつれて,公開したいタスクが増えていくと考えられる.
ただし,より具体的な下層の葉に近いタスクは,段階が進んだとしても公開に適さない場合が多いと考えられる.
そこで,タスクツリーからユーザが公開したいタスクのみを部分的に選択できるようなインターフェースも必要となる.
このような公開箇所の選択機構についても,現在実装中である.

\section{類似タスクの推薦機構}
TODO

\section{性能評価・考察}
TODO

\section{おわりに}
本稿では,タスクツリーを直感的ユーザインタフェースで編集可能であり,プロジェクトの段階を考慮したアクセス制御機構により部分的なオープンデータ化が可能なMissionForestの試作について述べた.
実装中であるアクセス制御機構および公開タスクの部分的選択機構については,実装が完了し次第,評価実験を行う予定である.


\begin{thebibliography}{9}
\bibitem{shiramatsu2016}
白松俊, Teemu Tossavainen, 大囿忠親, 新谷虎松: 社会課題とその解決目標のLinked Open Data化による目標マッチングサービスの開発. 人工知能学会論文誌, 31(1), pp. LOD-C\_1-11, (2016)
\begin{quote}
社会課題とその解決目標のLinked Open Data化による目標マッチングサービスの開発について述べている.
\end{quote}

\end{thebibliography}
%
%%%%%% 以下の行は消さないこと %%%%%%%
\clearpage                       %この行を消すと最終ページの枠線消滅の危機
\end{論文概要}                   %この行を消してはいけない
\end{document}                   %この行を消してはいけない
