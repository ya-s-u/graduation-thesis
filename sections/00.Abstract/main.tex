%!TEX root = ../../main.tex
\chapter*{論文要旨}
\label{chap0}
\addcontent{論文要旨}
大学の研究室では,教員の指示だけに頼らず主体的に考えて研究できる学生の育成が求められている.
しかし研究生活に慣れていない学生にとってはうまく研究が進まないことも多く,また教員も逐一フォローアップできていない現状がある.
そこで,「どのような課題をどのようなアプローチで解決しようとしているか」を日常的に学生自身がデータ化し共有することで,教員や他学生とのコミュニケーションを支援し,
協働を促進することができると考える.
学生の研究目標を公開・共有することによって,教員による進捗の把握や,学生の自律性向上,学生同士の協働の促進を目的としている.
また学外に対しても,研究アプローチや研究成果を公開することによって,外部組織との連携やアウトリーチ活動にも活用でき,公益活動にも研究シーズを活用できる可能性がある.

本研究ではこれまで,公益活動やシビックテックといった分野を対象とし,公共圏で目標を共有するWebシステム「ゴオルシェア」[1]を開発・運営してきた.
従来のゴオルシェアは組織横断的な協働を想定しており,目標データを全てオープンデータ化していた.
しかし,組織内での日常的な活動は公開に適さないものも多いため,日常的には使いにくいという問題点があった.
また,目標を階層化したツリー構造の入力操作が直感的でないという問題点もあった.
そこで本稿で試作する新システム「MissionForest」では,
(1)組織内部の日常的な活動を非公開な目標ツリーとして記録し,
(2)外部発表後にツリー構造の全体もしくは一部のみをLODとしてオープンデータ化可能にする.
さらに,(3)目標ツリーを直感的操作で作成・編集可能にする.
