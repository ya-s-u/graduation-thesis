%!TEX root = ../../main.tex
%修論用
% 参考文献 http://bibdesk.sourceforge.net/manual/BibDesk%20Help_94.html#SEC179 による出力
%以下テンプレ
%<$publications>
%\bibitem{<$citeKey/>}
%<$authors.name.stringByRemovingTeX.@componentsJoinedByComma/>: ``<$fields.Title.stringByRemovingTeX/>,''  <$fields.Journal.stringByRemovingTeX/>. Vol. <$fields.Volume/>,  No. <$fields.Number/>,  pp. <$fields.Pages/>,  <$fields.Year/>.
%<$fields.URL/>
%<$fields.localURL/>
%\begin{quote}
%\end{quote}

%
%</$publications>
\addcontent{参考文献}

\begin{thebibliography}{99}
% メディアについて

\bibitem{hatano2007}
波多野誼余夫(編):"音楽と認知" Vol. 8, 東京大学出版会, (2007)
\begin{quote}
様々な認知研究や音楽理論から音楽心理学を,人間はいかにして音楽を認知するかの研究として捉えて考察をのべている.
\end{quote}

\bibitem{songle}
能動的音楽鑑賞サービスSongle,\url{http://songle.jp/},2015年11月25日アクセス.
\begin{quote}
	能動的音楽鑑賞サービスSongleは動画サイトやネット上にアップロードされた合法的に視聴できる音楽を自動解析し,音楽と一緒にサビ,メロディ,コード,ビートを視覚的に表示させて鑑賞することができる.
\end{quote}

\bibitem{ongakunotomo1979}
音楽之友社:"音楽中辞典" 音楽之友社, (1979)
\begin{quote}
音楽用語の解説・用法を記載した辞典.
\end{quote}

\bibitem{suga2008}
菅道子. "身体表現を取り入れた参加型音楽コンサートの可能性: カノンの理解を目指した 「追いかけっこしよう」 の事例から" 和歌山大学教育学部教育実践総合センター紀要 18 , pp. 121-129, (2008)
\begin{quote}
	音楽理解における身体表現の有効性を小学校低学年を対象とした参加型音楽コンサートの企画・実施を通して検討し,旋律線のてなぞりなどの身体動作が音楽理解を促進することをあげている.
\end{quote}

\bibitem{takahasi2010}
高橋祐樹:"私の研究開発ツール Processing" 映像情報メディア学会誌 Vol. 64, No.12, pp. 1841-1849, (2010)
\begin{quote}
	Processingの説明と,基本的な使い方について述べている.
\end{quote}

\bibitem{nakamura2010}
中村俊介, 竹井将紫: "インタラクティブアート 「KAGURA」 によるワークショップ: 東京ミッドタウン・デザインハブ・キッズウィークにおける子供向けイベントの報告 (B-2 音楽と聴覚のデザイン, 研究発表, 芸術工学会 2010 年度秋期大会 in 浜松)."  芸術工学会誌,  Vol. 54, pp. 48-49, (2010)
\begin{quote}
カメラの前で動いて音声を生成するインタラクティブアート「KAGURA」をデジタルでしかできないインタラクティブな教育コンテンツとして利用しようと考え,東京ミッドタウンで開催された小・中学生向けワークショップで展示したときの報告が書かれている.
\end{quote}

\bibitem{kagura}
KAGURA,\url{http://www.kagura.cc/jp/},2015年11月25日アクセス
\begin{quote}
	Real Senseに対応した身体動作とジェスチャーで操作できる音楽演奏アプリケーション.
\end{quote}

\bibitem{kanke2013}
菅家浩之, 竹川佳成, 寺田努, 塚本昌彦:"Airstic Drum: 実ドラムと仮想ドラムを統合するためのドラムスティックの構築" 情報処理学会論文誌, Vol. 54, No. 4, pp. 1393-1401, (2013)
\begin{quote}
	楽器の運搬と演奏スペースの問題を解決するため,使用頻度の低い打楽器に仮想的なドラムを割り当て加速度センサの閾値から仮想ドラムの叩打と実ドラムの叩打を区別する方法について述べている.
\end{quote}

\bibitem{shiramatsu2015}
Shiramatsu, S., Ozono, T., and Shintani S.: A Computational Model of Tonality Cognition Based on Prime Factor Representation of Frequency Ratios and Its Application. Proc. of SMC 2015, (2015)
\begin{quote}
	調性理解モデルPFG Tonnetzについて述べている.
\end{quote}

\bibitem{behringer10}
Behringer, R. and Elliot, J.: Linking Phys- ical Space with the Riemann Tonnetz for Exploration of Western Tonality, chapter 6, pp. 131–143, Nova Science Publishers, (2010)
\begin{quote}
	1880年にRiemannが発展させたTonnetzを分析し,数式化した.
\end{quote}
%
%\bibitem{hewlett07}
%Hewlett, W., Selfridge-Field, E., and Cor- reia, E.: Tonal Theory for the Digital Age, Vol. 15 of Computing in Musicology, Center for Computer Assisted Research in the Humanities, Stanford University, (2007)
%\begin{quote}
%
%%\end{quote}
%
%\bibitem{tymoczko12}
%Tymoczko, D.: The Generalized Tonnetz, Journal of Music Theory, Vol. 56, No. 1, pp. 1–52, (2012)
%\begin{quote}
%
%\end{quote}

\bibitem{hand_labels}
Intel RealSense SDK 2015 R5 Documentation, \url{https://software.intel.com/sites/landingpage/realsense/camera-sdk/v1.1/documentation/html/index.html?jointtype_pxchanddata.html}, 2016年1月25日アクセス
\begin{quote}
	RealSenseが認識できる関節の種類が載っている.
\end{quote}

\end{thebibliography}
