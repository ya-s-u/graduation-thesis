%!TEX root = ../../main.tex
\chapter{まとめ}

\section{本研究の要約}
本稿では,
(1)組織内部の日常的な活動を非公開な目標ツリーとして記録し,
(2)外部発表後にツリー構造の全体もしくは一部のみをLODとしてオープンデータ化可能にする.
さらに,(3)目標ツリーを直感的操作で作成・編集可能にする,
ミッションフォレストの試作について述べた.
本システムを用いることで,タスク全体またはタスクの一部分を,個人的思想,組織内限定,外部公開,LODの4つの権限から設定することができ,
外部組織との連携やアウトリーチ活動にも活用でき,公益活動にも研究シーズを活用できる可能性がある.
実装中であるアクセス制御機構および公開タスクの部分的選択機構については,実装が完了し次第,評価実験を行う予定である.

本システムを大学の研究室で用いることにより,学生の研究目標を公開・共有することによって,教員による進捗の把握や,学生の自律性向上,学生同士の協働の促進を目的としている.

\section{今後の展望}
ユーザースキル推定アルゴリズムの精度を向上し,さらにユーザー同士のコラボレーションを促進するようなシステムにしたい.
また,ユーザースキルの推薦のみならず,類似ミッション及び類似タスクの推薦も可能にしたい.

来年度から研究室内で実際に使ってもらい,実運用をしながら改善していきたい.
また,研究室外のハッカソンなどのイベントで全く初めての人に使ってもらい,本研究の有効性を検証したい.

今後は,研究室内でのソースコード共有やハッカソンでの使用を想定し,GithubやSlackなどの外部サービスとの連携を可能にする.
またWebAPIを開発し,Web議論システムCOLLAGREEとの連携による協働プロセスのアーカイブ化など,
より実効性の高い協働支援のできるタスク構造化システムを開発していく予定である.
