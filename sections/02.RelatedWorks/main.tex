%!TEX root = ../../main.tex
\chapter{動作・開発環境と関連研究}
\section{動作・開発環境}
本研究の動作・開発環境について述べる.

\subsection{Ruby}
Ruby(ルビー)\footnote{https://www.ruby-lang.org}は,まつもとゆきひろにより開発されたオブジェクト指向スクリプト言語であり,スクリプト言語が用いられてきた領域でのオブジェクト指向プログラミングを実現する.
また日本で開発されたプログラミング言語としては初めて国際電気標準会議で国際規格に認証された事例となった[4].

\subsection{Ruby on Rails}
Ruby on Rails(ルビーオンレイルズ)\footnote{http://rubyonrails.org}は,オープンソースのWebアプリケーションフレームワークである.
その名にも示されているようにRubyで書かれている.
またModel View Controller(MVC)アーキテクチャに基づいて構築されている.

\subsection{MySQL}
MySQL(マイエスキューエル)\footnote{https://www.mysql.com}は,オープンソースで公開されている関係データベース管理システム (RDBMS) の一つである.

\subsection{Stardog}
Stardog(スタードッグ)\footnote{http://stardog.com}は,Javaで書かれた有償のRDFストア.
推論エンジンPalletを開発するなど,セマンティック技術開発分野での実績があるComplexible社に夜製品.
特徴として,OWL2での推論や,ルール言語SWRLに対応している.
また,セキュリティ関連機能を持ち,データベースとユーザーそれぞれについてアクセス権を設定できる.
API経由での利用も可能.

\section{関連研究}
本研究と関連のある研究やシステムを紹介する.

\subsection{ゴオルシェア}
ゴオルシェア\cite{shiramatsu2016}とは,社会課題とその解決目標,さらにその部分目標をLOD化して共有するためのWebサービスです.
方向性の似た潜在的協力者を探せるようになります.
さらに有力者の目標も共有し,近未来社会像の透明性向上を図ります
