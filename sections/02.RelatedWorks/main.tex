%!TEX root = ../../main.tex
\chapter{動作・開発環境と関連研究}
本章では本研究の動作・開発環境と関連研究について述べる.

\section{動作・開発環境}

\subsection{Ruby}
Ruby(ルビー)\footnote{https://www.ruby-lang.org}は,まつもとゆきひろにより開発されたオブジェクト指向スクリプト言語であり,スクリプト言語が用いられてきた領域でのオブジェクト指向プログラミングを実現する.
また日本で開発されたプログラミング言語としては初めて国際電気標準会議で国際規格に認証された事例となった[4].

\subsection{Ruby on Rails}
Ruby on Rails(ルビーオンレイルズ)\footnote{http://rubyonrails.org}は,オープンソースのWebアプリケーションフレームワークである.
その名にも示されているようにRubyで書かれている.
またModel View Controller(MVC)アーキテクチャに基づいて構築されている.

\subsection{MySQL}
MySQL(マイエスキューエル)\footnote{https://www.mysql.com}は,オープンソースで公開されている関係データベース管理システム (RDBMS) の一つである.

\subsection{Stardog}
Stardog(スタードッグ)\footnote{http://stardog.com}は,Javaで書かれた有償のRDFストア.
推論エンジンPalletを開発するなど,セマンティック技術開発分野での実績があるComplexible社に夜製品.
特徴として,OWL2での推論や,ルール言語SWRLに対応している.
また,セキュリティ関連機能を持ち,データベースとユーザーそれぞれについてアクセス権を設定できる.
API経由での利用も可能.

\section{関連研究}
本研究と関連のある研究やシステムを紹介する.

\subsection{タスク階層構造}
// TODO:

\subsubsection{work breakdown structure}
WBSとは,プロジェクトマネジメントで計画を立てる際に用いられる手法の一つで,プロジェクト全体を細かい作業に分割した構成図.
WBSでは,まずプロジェクトの成果物をできるだけ細かい単位に分解していく.
その際,全体を大きな単位に分割してから,それぞれの部分についてより細かい単位に分割していき,階層的に構造化していく.

\subsubsection{前提条件ツリー}
前提条件ツリーの構成要素には,実現すべき目標と中間目標,そして目標を実現する上での障害がある.前提条件ツリーで扱う関係は必要関係と克服関係である.
必要関係では,上位の目標を実現するためには下位の目標が必要であることを表す.これに対して克服関係では上位の目標の障害を下位の中間目標によって解決できることを表す.
新たなシステムが成功するための前提条件を予め明らかにするために作成するのが前提条件ツリー

\subsection{プロジェクト管理システム}
// TODO:

\subsubsection{ゴオルシェア}
ゴオルシェア\cite{shiramatsu2016}とは,社会課題とその解決目標,さらにその部分目標をLOD化して共有するためのWebサービスです.
方向性の似た潜在的協力者を探せるようになります.
さらに有力者の目標も共有し,近未来社会像の透明性向上を図ります

\subsubsection{ナレッジコネクター}
「Knowledge Connector」は,全国的に行われているオープンデータを活用したイベントの成果等を集約し,一元的に検索を可能にするとともに,アイディアやアプリを創出した人材とビジネスパートナーとのマッチングを支援するためのサイト.

\subsubsection{ミライラボ}
「みらいらぼ」という市民共創プロジェクト支援システムを採用している.
これを用いると,学会開催以前からプロジェクトの概要を共有し,おおまかな進捗管理をすることができる.
また,プロジェクト参加者の日々のアウトプットやコミュニケーションのための機能として,ツイートのような短文投稿機能も開発中である

\subsection{合意形成システム}
// TODO:

\subsubsection{COLLAGREE}
合意形成システム「COLLAGREE」とは,離れた場所にいる人々がweb上での議論を通して合意形成をはかるシステムです.
皆さんのつぶやきがシステムによって意見集約され,合意形成を行います.
限られた人々が一室に集まって行う従来の閉じられた議論ではなく,多くの人々がいつでもインターネット上で開かれた議論を行えることを目指しています.
