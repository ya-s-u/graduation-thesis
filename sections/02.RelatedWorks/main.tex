%!TEX root = ../../main.tex
\chapter{関連研究と動作・開発環境}
本章では本研究の関連研究と動作・開発環境について述べる.

\section{関連研究}
本研究と関連のある研究やシステムを紹介する.

\subsection{タスク階層構造}

\subsubsection{Work Breakdown Structure}
Work Breakdown Structure(WBS)とは,プロジェクトマネジメントで計画を立てる際に用いられる手法の一つで,プロジェクト全体を細かい作業に分割した構成図.
「作業分割構成」,「作業分解図」などとも呼ばれる.
WBSでは,まずプロジェクトの成果物をできるだけ細かい単位に分解していく.
その際,全体を大きな単位に分割してから,それぞれの部分についてより細かい単位に分割していき,階層的に構造化していく.
青果物の細分化が終わったら,それぞれの部分を構成するのに必要な作業を考え,最下層に配置していく.

\subsubsection{前提条件ツリー}
前提条件ツリー(Prerequisite Tree)\cite{zentei}の構成要素には,実現すべき目標と中間目標,そして目標を実現する上での障害がある.
前提条件ツリーで扱う関係は必要関係と克服関係である.
必要関係では,上位の目標を実現するためには下位の目標が必要であることを表す.
これに対して克服関係では上位の目標の障害を下位の中間目標によって解決できることを表す.
新たなシステムが成功するための前提条件を予め明らかにするために作成するのが前提条件ツリーである.

\subsection{プロジェクト管理システム}

\subsubsection{ゴオルシェア}
ゴオルシェア\cite{shiramatsu2016}とは,社会課題とその解決目標,さらにその部分目標をLOD化して共有するためのWebサービスである.
「誰がどんな課題に注目しているのか」「誰が何を目指して動いているのか」を一般市民が自ら入力して共有することで,
方向性の似た潜在的協力者を探すことができ,さらに有力者の目標を共有し,近未来社会像の透明性向上を図る.
LODの枠組みによって,社会課題と解決目標の関係や,その部分目標との関係,協力者との関係などを紐付けし,似た目標を探せるようにした.
多くの利害関係者が絡む社会問題を解決するためには,組織横断的な協働が不可欠であり,
従来のSNSには方向性の似た潜在的協力者を探す機能がなかったので,このようなシステムがあれば,
(1) 身の回りの社会課題や目指す目標を一般市民が公開&共有して,似た方向性の人々を探せるようにする,
(2) 断片的なニュースから,社会課題や有力者の目標を抽出&共有して,世の中の動きの透明性を向上させる,
ことができる,と述べられている.

\subsubsection{ナレッジコネクター}
ナレッジコネクター(Knowledge Connector)は,全国的に行われているオープンデータを活用したイベントの成果等を集約し
,一元的に検索を可能にするとともに,アイディアやアプリを創出した人材とビジネスパートナーとのマッチングを支援するためのサイト.
同じような課題を抱えている人,何かをやりたいと思っているが何から手をつけていいかわからない人,もしかしたら連携できる人やコミュニティを見つけることができる.

\subsubsection{みらいらぼ}
みらいらぼ\cite{sengoku2016}とは,共創の場を支援する集合知プロジェクト支援システムである.
みらいらぼは,Webアプリケーションであり,Web上でプロジェクトの作成,共有,およ ゙共創をすることができる.
また,みらいらぼはSNSとの連携が可能であり,多くの多様な人たちとプロジェクトの共有を行うことができ,集合知を用いてプロジェクトを進めることができる.
さらに,Web上だけでなく学会と合わせてみらいらぼを使うことによって,より共創を起こすことが可能である.

\subsubsection{企業内情報の有効活用}
年々増加傾向にある企業内情報をシステム横断的に有効活用できる仕組みが求められている.
業務とその一連のプロセスをタスクと呼ぶこととし,企業内の様々なシステム上の情報をタスクという観点で活用できるようにするために,
タスクに関するメタデータを付与する手法\cite{metadata}を提案する.
どのようなタスクがあるのかをルールで抽出し,タスクにおいて作成及び活用された企業内情報を,
ルールと機械学習を組み合わせて抽出を行う.
提案手法を企業内の実データに対して適用検証を行った結果が報告されている.

\subsubsection{人間関係を用いたシステム}
複数の個人による協調的なタスク管理手法モデル\cite{scheduler}を提案し,これを実現するためのアプリケーションとして携帯電話用スケジューラの実装を行った.
一般に,複数人がかかわる共同作業のようなタスクの管理コストを下げるためには,参加者がスケジュール情報を公開することが望ましい.
しかしながら多くの個人は複数の組織に所属しており,すべての情報を公開することは難しい.
しかも,それらの組織の一部はアドホックに形成されるためにグループウェアに代表されるトップダウンの管理モデルを適用することは難しい.
そこで,個人同士がタスクの依頼及び受理を行った履歴より人間関係ネットワークを生成し,
完全グラフとなる部分グラフをグループとみなす手法及び発見されたグループ単位での情報のアクセスコントロールを提供する手法によって,
プライバシー侵害の問題を低減した上で情報公開を行うことを可能にする.
また,明示的に記述されたプロファイルを必要としないために,ユーザの情報入力の付加を軽減することができる.
これらの手法を日常的なスケジューリングに適用するために携帯電話アプリケーションを開発し,
実証実験を行った結果,半自動的なグループ発見及びアクセスコントロールは有効であるとの結論を得た,と述べられている.

\subsubsection{目標マネジメント}
現在の大学生は授業や部活動などで複数の活動に並行して参画しており,常に多くの作業を抱えている.
そのため,現在,学生が主体となって問題解決のプロジェクトを実施する形式の授業であるPBL(Project BasedLearning)型授業においても,
学生がスケジュール調整やプロジェクトの遂行計画立案ができずに,プロジェクトの目標が達成できないなどの問題が生じている.
そこで,企業活動で注目されているプロジェクトマネジメントを体系的に考える手法に着目し,
プロジェクトに関して専門知識や豊富な経験を持たない学生でも,プロジェクトを円滑に実施できるようコンピュータが学生を支援する方法\cite{management}を検討する.
このような,支援により,学生同士のスケジュール調整の効率やプロジェクトの遂行計画の共通理解を促進し,
プロジェクトメンバの授業への参加意欲やチームワーク,グループでの計画遂行能力の学習効果の向上につなげたい.

\subsection{合意形成システム}

\subsubsection{COLLAGREE}
COLLAGREE\cite{ito2016}とは,離れた場所にいる人々がweb上での議論を通して合意形成をはかるシステムである.
投稿されたつぶやきがシステムによって意見集約され,合意形成を行います.
限られた人々が一室に集まって行う従来の閉じられた議論ではなく,多くの人々がいつでもインターネット上で開かれた議論を行えることを目指している.

\subsubsection{Community Organizer}
ネットワーク上でのコミュニティの形成を支援するシステムCommunity Organizer\cite{organizer}において,人々の興味や関心及び人々の間で交わされる情報を,
それらの関連性に基づいて空間的に表現してユーザに提示する情報提示手法を提案するとともに,
その有効性を検証するために行った対照実験の結果について述べられている.
Community Organizerが支援の対象として注目するのはネットワーク上のコミュニティの初期段階である.
参加者や話題が流動的に変化する集団を対象として,境界があいまいな集団におけるコミュニケーションを支援し,
参加者がそこにコミュニティを自発的に感じとれるようなインタラクションを生じさせることがシステムの目標である.
ここで提案する情報提示手法がコミュニティ形成に与える影響を検討するため,
我々はその機能をもつものともたないものの2種類のソフトウェアを作成して対照実験を行い,
Community Organizerが提供する空間的表現を用いた情報提示手法が有効であることを確認した.

\subsection{LinkedData}

\subsubsection{作成支援ツール}
LinkedData作成支援ツールには,大きく分けて生データからLinkedDataを作成するツールと,
新しくデータを入力することでLinkedDataを作成するツールがある\cite{tool}.
LinkedDataを作るにはプログラミングが必要であるとなると,元々公開システムを作っているところであれば対応できるかもしれないが,
そうではない一般のユーザには取り組みにくい.
生データを作成しているユーザは,Excelのようなスプレッドシートや関係データベースでデータを管理している場合が多いため,
これらから手軽に LinkedDataを作成して提供するための支援ツールが,LinkedDataの普及に重要な役割を持つと考えていると述べられている.

\subsubsection{企業での活用}
知識集約型組織では,経営戦略上の各種の意思決定に有用な知識や洞察を生み出すために,企業内外の事実に基づく情報を組織的にかつ系統的に蓄積・分析し,それを活用することが求められている.
一つの例として研究所がどんな研究をしているのかを見える化し,営業・SEが研究所の技術を探せる/発見があるようにするために,
社内外の文書,展示会,人物,組織,技術,研究,ソリューション等に関するデータをLinkedData化するとともに,
探索の効率化のためのネットワーク表示機能,知識を増殖したり維持管理するための集合知活用などを取り入れた、
このLinkedDataを活用するツールを作成した\cite{company},と述べられている.

\section{動作・開発環境}

\subsection{Ruby}
Ruby\footnote{https://www.ruby-lang.org}は,まつもとゆきひろ氏により開発され,1995年12月にfj上で発表されたオブ
ジェクト指向スクリプト言語である.機能として,クラス定義,ガベージコレクション,
強力な正規表現処理,マルチスレッド,例外処理,イテレータ・クロージャ,Mixin,演
算子オーバーロードなどがある..Rubyにおいては整数や文字列なども含めデータ型は
すべてがオブジェクトであり,純粋なオブジェクト指向言語といえる.

\subsection{Ruby on Rails}
Ruby on Rails\footnote{http://rubyonrails.org}はデンマークのDavid Heinemeier Hanssonにより作成されたフレーム
ワークで,2004年7月に公開され,2005年12月13日にバージョン1.0がリリースされた.
Ruby on Railsは,フレームワークとしては一般的なMCV型に属する.MCV型とは,
Model-View-Controllerの頭文字をとった言葉で,データとそれに関わる処理を担う「モ
デル(Model)」,表示・出力を行う「ビュー(View)」,これらのビューとモデルを制御す
る「コントローラー(Controller)」といった3つのコンポーネントを基礎とするアーキテ
クチャである.このように機能を分割しておくと,ビュー・コントローラーとモデルが疎
結合になることによってモデルの再利用性が高まり同時にビューの入れ替えが容易にな
り,ソースコードの見通しもよくなる.
Ruby on Railsには,基本理念として「同じことを繰り返さない」(DRY:Don't Repeat
Yourself)と「設定よりも規約」(CoC:Convention over Configuration)がある.「同じこ
とを繰り返さない」(DRY)というのは,Andrew Hunt氏とDevid Thomas氏が打ち出した
原則で,定義などの作業は一回だけですませ,冗長なだけでなく往々にして間違いの元に
なる,重複や繰り返しを避けろという意味である.「設定よりも規約」(CoC)とは,Ruby
on Rails発祥の哲学で,レアなケースをも想定して冗長になりがちな設定を極力排除し,
規約に従うことでお決まりの動作を実現させようという思想である.
また,Ruby on Railsはフレームワークであるだけでなく開発環境でもある.Ruby on
Railsには,コード生成機能があり,基本理念の1つである「設定よりも規約」に基づく
テンプレートが配置され,通常1画面ごとに手作業で用意していたスケルトンがスクリプ
ト1つで生成でき,また,データベースレコードの新規作成,参照,更新,削除を行うよ
うな単純なWebアプリケーションもまたスクリプト1つで生成できる.さらに,Ruby on
Railsに用意されているのはコード生成機能だけではなく,すぐに動かして試せる簡易
Webサーバやインタラクティブシェルといったスクリプト,コード生成時にはテストコー
ドのひな型も同時に生成さる.これらにより,Webアプリケーションの開発が容易にな
る.

\subsection{MySQL}
MySQL(マイエスキューエル)\footnote{https://www.mysql.com}は,オープンソースで公開されている関係データベース管理システム (RDBMS) の一つである.
データベースでデータを操作する際には,「SQL」という言語を使う.
データの追加,削除,変更,検索などを行うだけでなく,データベースの構造やデータ型を定義する言語でもある.
他には PostgreSQL,Oracle などがある.
MySQLは他言語と比べ,処理速度が速く,非常に多くのデータを扱えるので,広い範囲での応用が可能である.
また,C,Java,Perl,PHP,Ruby など様々なプログラミング言語と接続可能で,多言語対応で,日本語は,SJIS,EUC ともにサポートしている.

\subsection{Stardog}
Stardog(スタードッグ)\footnote{http://stardog.com}は,Javaで書かれた有償のRDFストア.
推論エンジンPalletを開発するなど,セマンティック技術開発分野での実績があるComplexible社による製品.
特徴として,OWL2での推論や,ルール言語SWRLに対応している.
また,セキュリティ関連機能を持ち,データベースとユーザーそれぞれについてアクセス権を設定できる.
API経由での利用も可能である.
