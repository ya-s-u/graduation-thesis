%!TEX root = ../../main.tex
\chapter{序論}
\label{chap:intro}

\section{本研究の背景}
大学の研究室では,教員の指示だけに頼らず主体的に考えて研究できる学生の育成が求められている.
しかし研究生活に慣れていない学生にとってはうまく研究が進まないことも多く,また教員も逐一フォローアップできていない現状がある.
そこで,「どのような課題をどのようなアプローチで解決しようとしているか」を日常的に学生自身がデータ化し共有することで,教員や他学生とのコミュニケーションを支援し,
協働を促進することができると考える.

\section{本研究の目的}
学生の研究目標を公開・共有することによって,教員による進捗の把握や,学生の自律性向上,学生同士の協働の促進を目的としている.
また学外に対しても,研究アプローチや研究成果を公開することによって,外部組織との連携やアウトリーチ活動にも活用でき,公益活動にも研究シーズを活用できる可能性がある.

このようなシステムに必要になる要件は,(1)プロジェクト全体のタスクを俯瞰できる,(2)各タスクの進捗状況が把握できる,(3)各タスクに対して議論することができる,という3つを満たす必要がある.
このような要件を満たすシステムは”プロジェクト管理システム”と呼ばれ,有償無償問わず数多く存在する.
上記のような従来型のプロジェクト管理システムの機能に加え,プロジェクトの成果を公開することで,2次活用や外部組織との連携に役立てられることができれば,新たな協働を生む可能性がある.
そこで新システムMissionForestの試作により,新たな協働を生み,協働を支援できるようなプロジェクト管理システムを目指す.
プロジェクトの目標階層は,ゴオルシェアを踏襲してLinked Dataとして構造化した上で,ゴオルシェアよりも詳細に公開/非公開を制御できる機構を目指す.

\section{本論文の構成}
本論文の構成を以下に示す.
第2章では関連研究や関連システムと本研究の動作・開発環境について述べる.
第3章では直感的にツリーを作成できるインターフェースについて述べる.
第4章ではLODでの公開機構について述べる.
第5章では類似タスクの推薦機構について述べる.
第6章では前章までで述べたシステムの評価実験の結果をまとめる.
第7章で本研究をまとめる.
